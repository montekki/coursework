\bigskip
\section {Заключение} 

В процессе решения поставленной задачи 
с использованием результатов работы прошлого 
года была разработана система контроля 
поведения приложения, различающая его 
внутренние состояния и корректирующая
относительно изменений внутренних состояний
ограничения, накладываемые системой SELinux
на поведение приложения. Данная система была 
реализована в качестве модуля ядра. При этом, 
изменения в самом ядре Linux являются минимальными, 
а утилиты пользовательского пространства, входящие
в проект SELinux остаются неизменными. Это позволяет 
легко поддерживать решение в процессе выхода новых 
версий ядра Linux. 

Кроме этого, реализованное решение достаточно 
слабо зависит от SELinux, используя лишь функции 
изменения контекста приложения в процессе исполнения.
Это позволит в случае необходимости легко 
перенести реализованную систему на другие системы 
контроля поведения приложений, существующие в Linux
такие, как AppArmor и Tomoyo.

Исполнимый код наблюдаемого приложения размечается 
контрольными точками, которые используются для 
разграничения пространства его внутренних состояний. 
Расставленные в коде контрольные точки не могут быть 
изменены или удалены злоумышленником, как не могут 
быть добавлены и новые контрольные точки. Это позволяет 
считать информацию о смене состояний надежной. 

На данный момент из-за высокой технической сложности 
не удалось привести требуемый пример уязвимого приложения 
или создать такой пример искуственно. Такой пример 
крайне желателен для обоснования актуальности проделанной
работы и дальнейшие усилия будут сконцентрированы именно 
на этом. 


\bigskip
\begin{thebibliography}{99}
\bibitem{SEOF} 
Официальная документация SELinux [HTML] 

(http://www.nsa.gov/research/selinux/docs.shtml)
\bibitem{AppArmor} 
Документация по проекту AppArmor [HTML]

(http://en.opensuse.org/AppArmor\_Geeks)
\bibitem{pax} 
Сайт проекта GRSecurity [HTML] 

(http://pax.grsecurity.net/)

\bibitem{LDD}
Jonathan Corbet, Greg Kroah-Hartman, Allesandro Rubini Linux Device Drivers, O'Reilly, 2005. 640 c.

\bibitem{ULK} 
Daniel P. Bovet, Marco Cesati, Understanding the Linux Kernel, O'Reilly, 2002, 568 c.
 
\bibitem{utrace} 
Страница utrace [HTML] 

(http://people.redhat.com/roland/utrace/)

\bibitem{gornak}
Горнак Т.А. Инструментальное средство автоматизации построения моделей
нормального поведения приложений. МГУ, 2010.

\end{thebibliography}
