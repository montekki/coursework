\bigskip
\subsection{AppArmor}

AppArmor не использует
явную разметку всех объектов в системе.
AppArmor контролирует поведение 
приложений, опираясь на профили поведения
приложений, описанные на некотором интерпретируемом
языке. 
В данных файлах хранится информация, основанная на 
путях к объектам в файловой системе, о том, к каким 
объектам и с какими правами имеет доступ приложение. 

В отличие от SELinux, 
в которой настройки глобальны для всей системы, профили 
AppArmor разрабатываются индивидуально для каждого 
приложения.
Таким образом, гораздо меньше вероятность 
необходимости изменения существующих профилей при 
генерации новых профилей. Кроме этого, 
AppArmor предоставляет инструменты 
автоматической генерации профилей на основе поведения 
приложения и возможность производить контроль в двух 
режимах: режиме обучения и режиме принуждения. 

\begin{comment}
AppArmor является системой безопасности, поддерживаемой 
компанией Novell, включена в дистрибутивы openSUSE и SUSE 
Enterprise.  Изначально в AppArmor включен набор 
стандартных профилей, запускаемых после установки. 
Отдельно доступны профили для разных популярных программ 
и серверов. Кроме этого, существуют инструменты для генерации 
профилей (genprof и logprof). Основная идея — верный выбор 
приложений, нуждающихся в ограничении привилегий и 
создание/редактирование профилей безопасности. Таким 
образом, в случае эксплуатации злоумышленником 
уязвимости,  нанесенный ущерб сводится 
к минимуму. 

Система может работать в двух режимах: режиме 
обучения (complain) и в принудительном режиме (enforce). 
В первом из них все нарушения правил профиля разрешены, 
но немедленно регистрируются. Загрузка профиля в 
принудительном режиме предписывает системе отправлять 
сообщения о нарушениях в syslogd. Запуск и остановку 
AppArmor можно осуществлять при помощи команды rcapparmor 
с одним из следующих параметров: start (загрузка модуля 
ядра, анализ профиля, монтирование своей фс); stop (фс 
размонтируется, профили становятся недействительными); 
reload (перезагрузка профилей), status (информация о 
количестве запущенных профилей, в каком режиме они 
работают). Инструменты командной строки AppArmor: autodep 
(создает приблизительный профиль для программы или 
рассматриваемого приложения); complain (устанавливает 
профиль AppArmor в обучающий режим); enforce (переводит 
профиль в принудительный режим); genprof (генерирует профиль,
программа указывается при запуске); logprof (управляет
профилями AppArmor); unconfined (выводит список процессов 
с портами tcp и udp, которые не имеют загруженных 
профилей AppArmor). Система AppArmor построена на системе 
полных путей к файлам, проще говоря, типичное описание 
профиля выглядит примерно так: 

\bigskip
\begin{lstlisting}
#include <tunables/global>  
/usr/bin/man { 
	#include <abstractions/base>
	#include <abstractions/nameservice>
	capability setgid,
	capability setuid,
	/usr/lib/man-db/man Px,} 
}
\end{lstlisting}
\end{comment}

\bigskip
Профиль состоит из записей, которые содержат 
полные пути до файлов или каталогов 
каталогов с указанием полных и указания прав
доступа к ним. При этом r — 
разрешение на чтение, w — запись(за исключением создания 
и удаления файлов), ix — исполнение и наследование текущего 
профиля, px — исполнение под специфическим профилем, Px — 
защищенное выполнение, ux — неограниченное исполнение, 
Ux — защищенное неограниченное исполнение, m — присвоение 
участку памяти атрибута «исполняемый», I — жесткая ссылка. 
Чтобы подключить готовый профиль к AppArmor, достаточно его 
скопировать в каталог /etc/apparmor.d. 

\bigskip
{\bfseries Методы контроля 
за внутренним состоянием приложения} 

Система предоставляет возможность смены
текущих привилегий для веб-сервера Apache
(Change Hat). Тем не менее, из-за фактического
прекращения разработки AppArmor данная 
система так и не стала доступной для использования
с произвольным приложением. 



\bigskip
\subsection{PaX} 

PaX - набор патчей ядра от GRsecurity. 

PaX может рассматриваться как дополнение к 
таким системам безопасности, как SELinux и 
AppArmor. 
Существует три класса угроз, предотвращением которых 
занимается PaX. Это внедрение и исполнение кода с 
повышенными привилегиями, исполнение кода самого 
процесса путем изменения нормального течения 
исполнения процесса, нормальное исполнение программы, 
но над данными, для которых предусмотрены повышенные 
привилегии. Non-executable pages (NOEXEC) и mmap/mprotect 
(MPROTECT) предотвращают внедрение и исполнение 
кода с повышенными привилегиями. 
Address Layout Randomisation (ASLR) 
позволяет предотвратить все три упомянутые вида атак в той 
ситуации, когда атакующий заранее закладывается на 
адреса в атакуемом процессе и не может узнать о них 
в процессе исполнения. Так как PaX полностью внедрен 
в ядро, предполагается то, что ядро является Trusted 
Computer Base. Инструментарий позволяет предотвратить 
исполнение стека, обеспечить рандомизацию 
размещения адресов внутри адресного пространства 
(address space layout randomization) . 
Основная цель данного проекта — изучение различных защитных 
механизмов, защищающих от эксплойтов уязвимостей ПО, которые 
предоставляют злоумышленнику полные права на чтение/запись в 
системе. Исполнение кода связано с необходимостью изменять 
ход выполнения процесса используя уже существующий код. Одна 
из основных проблем — подмена адресов возврата из функций и 
подмена самих адресов функций. Для установки PaX требуется 
наложить патч на дерево исходных кодов ядра, после чего собрать 
ядро и установить в систему. 

\subsection{Trusted BSD} 

Проект TrustedBSD – проект разработки 
расширения существующей системы 
безопасности FreeBSD, который включает 
в себя  расширенные атрибуты UFS2, 
списки контроля доступа, OpenPAM, аудит событий 
безопасности с OpenBSM, мандатное управление доступом 
и TrustedBSD MAC Framework. 
Расширенные атрибуты UFS2 позволяют ядру и 
пользовательским процессам помечать файлы 
именованными метками. В этих метках
хранятся данные, необходимые системе безопасности. 
ACL и метки MAC в их числе. Списки контроля доступа~--- 
 расширения дискреционного контроля доступа. Аудит 
системных событий позволяет вести избирательный 
логгинг важных системных событий для последующего 
анализа, обнаружения вторжений, и мониторинга  
. Начиная с версии 5.0 в ядре FreeBSD 
появилась поддержка MAC Framework, прошедшая испытания 
в TrustedBSD. Данный фреймворк позволяет создавать политики, 
определяющие принудительное присвоение доменов и типов (DTE), 
многоуровневую систему безопасности (MLS). Данный фреймворк 
предоставляет интерфейсы управления фреймворком, примитивы 
для синхронизации, механизм регистрации политик, примитивы 
для разметки объектов системы, разные политики, 
реализованные в виде модулей политики MAC и набор 
системных вызовов для приложений. При регистрации 
политики, происходит регистрация специальной структуры 
(struct mac\_policy\_ops), содержащей функции MAC 
framework, реализуемые политикой. На данный момент 
существуют следующие политики: 

mac\_biba – Реализация политики Biba, во многом 
схожей с MLS. Позволяет присваивать объектам и 
субъектам системы атрибуты доступа, которые образую 
иерархию уровней. Все операции над информацией в 
системе контролируются исходя из уровней 
взаимодействующих сущностей. 

mac\_ifoff позволяет администраторам контролировать 
сетевой трафик. 

mac\_lomac (Low-watermark MAC) еще одна 
реализация многоуровневого контроля доступа. 

mac\_bsdextended (file system firewall ) Система 
защиты файлов, основанная на определении прав 
доступа на основании роли пользователя. 

mac\_mls~--- реализация политики MLS. Объекты 
классифицируются  некоторым образом, субъектам 
присваивают уровень доступа. 

{\bfseries Предоставляемые системой методы контроля за 
внутренним состоянием приложения} 

Фреймворк MAC позволяет реализовать в модуле 
безопасности возможность изменения приложением 
собственных прав. 

\begin{comment}
\subsection{Недостатки существующих методов контроля 
за внутренним состоянием приложения} 

Основной идеей существующих методов является 
предоставление приложениям определенных интерфейсов
для изменения собственного состояния. Это является 
очередным шагом в сторону более безопасного 
программирования и альтернативой рекомендуемому 
подходу, который заключается в разбиении приложения 
на несколько более мелких приложений, которые 
соответствовали бы внутренним состояниям 
исходного приложения. 
При этом, изменения прав могут осуществляться
традиционно на основании наследования доменов во время 
execx(). 

Основным недостатком данного подхода является 
отсутствие возможности отличить 
вызовы функций смены доменов, внедренных в код 
приложения разработчиком, от вызова 
аналогичных функций злоумышленником в результате 
изменения нормального хода исполнения программы.
Кроме этого, данный подход предполагает, что 
разработчики сами должны определять необходимые 
места для внедрения таких вызовов, что не всегда 
возможно в силу ряда причин. В этом случае, 
изменения должны производиться третими 
лицами, как следствие появляются патчи, зависимые 
от версии, плюс необходимость пересобирать 
приложение с выходом каждой новой версии. 
\bigskip
\end{comment}
