
\bigskip
\subsection{Удобство использования}

\bigskip
{\bfseries SELinux.} Конфигурация политик является 
весьма сложной задачей, учитывая наличие специального
языка написания политик, сложности в написании правил.
Кроме этого, добавление новых профилей может повлечь 
за собой необходимость в модификации уже имеющихся 
профилей, отрицательно сказываясь на удобстве и 
простоте описания политик. 

\bigskip
{\bfseries AppArmor.} Гораздо меньше вероятность 
необходимости изменения существующих профилей при 
генерации новых профилей. Правила для приложений 
основываются на путях к файлам, что упрощает понимаие 
политик и не требует производить перепомечание всех 
объектов в системе. 

\bigskip
{\bfseries Vista.} Не актуально. 

\bigskip
{\bfseries PaX.} Наложение патча на ядро. 

\bigskip
{\bfseries TrustedBSD.} Определение удобства 
использования требует дополнительных исследований. 

\bigskip
{\bfseries ОС Феникс.} На данный момент находится
в стадии разработки и обладает инструментами для 
администрирования системы из среды ОС Windows. 

{\bfseries Apple Seatbelt} — Использование готовых 
конфигураций, либо создание собственных. Языком 
конфигурации является Scheme-подобный язык, сложный 
для восприятия и описания профилей. Официально это 
решение пока отсутствует в Mac OS X.

\bigskip
\begin{tabular}{|p{2cm}|p{4cm}|p{4cm}|p{4cm}|}
\hline
%\multicolumn{4}{|c|}{} \\
Система & Угрозы & Возможности & Удобства \\
\hline
\ttfamily SELinux 
& 
Предотвращение компрометации системы путем 
ограничения поведения 
& 
Возможность создания пользовательских ролей и 
контекстов безопасности 
& 
Не очень удобен в плане устновки и настройки, 
сложный язык описания политик.\\
\hline 
\ttfamily AppArmor
&
Аналогично SELinux
&
Четкое описание объектов, к которым может иметь 
доступ то или иное приложение и прав доступа.
&
Язык описания профилей интуитивно понятен, 
есть возможность автоматической генерации профилей \\
\hline 
\ttfamily PaX
&
Позволяет предотвратить угрозы, в которых
злоумышленник атакует некоторые уязвимости в системе
&
Защита памяти от исполнения, рандомизация адресов в программе.
&
Патч ядра \\ 
\hline 
\ttfamily Vista
& 
Модификация системных структур и кода ядра, уязвимости 
драйверов, прямой доступ к памяти
&
Предоставляет более широкие возможности для 
настройки и обеспечения безопасности системы, 
чем более ранние версии
&
Ничего не нужно делать\\
\hline
\ttfamily TrustedBSD
&
Предоставляет собой повышенный уровень защиты 
для соответствия стандартам «оранжевой книги»
&
Предоставляют более широкие возможности для 
настройки и обеспечения безопасности системы, 
чем стандартные версии 
&
Затраты на полное администрирование всей системы\\
\hline
\ttfamily Seatbelt
& 
Предоставляет возможности определения ограничений на 
поведение приложения, в том числе и помещение 
приложений в «песочницу»
&
Предположительно предоставляет более широкие возможности 
для настройки безопасности системы,  но все еще находится 
в стадии разработки. Отсутствуют порты модулей из 
родственной TrustedBSD
&
Сложный язык конфигурации политики, отсутствие 
документации. \\
\hline
\end{tabular}

\newpage
