
\bigskip 
{\bfseries Существующий в SELinux метод динамического 
переключения контекста} 

В 2004-м году в SELinux началась работа над системой 
интерфейсов для пользовательских приложений, позволяющих
приложениям динамически изменять свой контекст. Данная 
система предполагает, что приложение должно быть тесно 
интегрировано с существующей политикой и в зависимости 
от своего текущего состояния сообщать SELinux о смене 
контекста. Такой подход позволил бы создавать более 
безопасные приложения, при разработке которых возможно
было бы выделять состония, в которых приложению нужны
различные минимальные права. Такими интерфейсами стали 
функции 

\bigskip 
\begin{lstlisting} 
#include <selinux/selinux.h>

	int getcon(security\_context\_t *context);

	int getprevcon(security\_context\_t *context);

	int getpidcon(pid\_t pid, security\_context\_t *context);

	int getpeercon(int fd, security\_context\_t *context);

	int setcon(security\_context\_t context);
\end{lstlisting}
